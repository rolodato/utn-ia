\documentclass[a4paper,12pt]{article}
\usepackage[a4paper]{geometry}
\usepackage[T1]{fontenc}
\usepackage{lmodern}
\usepackage[english,spanish]{babel}
\usepackage[utf8]{inputenc}
\usepackage{hyperref}
\usepackage{multicol}

\usepackage{color}
\definecolor{dkgreen}{rgb}{0,0.6,0}
\definecolor{gray}{rgb}{0.5,0.5,0.5}
\definecolor{mauve}{rgb}{0.58,0,0.82}
\usepackage{listings}
\lstdefinelanguage{scala}{
  morekeywords={abstract,case,catch,class,def,%
    do,else,extends,false,final,finally,%
    for,if,implicit,import,match,mixin,%
    new,null,object,override,package,%
    private,protected,requires,return,sealed,%
    super,this,throw,trait,true,try,%
    type,val,var,while,with,yield},
  otherkeywords={=>,<-,<\%,<:,>:,\#,@},
  sensitive=true,
  morecomment=[l]{//},
  morecomment=[n]{/*}{*/},
  morestring=[b]",
  morestring=[b]',
  morestring=[b]"""
}
\lstset{literate=
	{á}{{\'a}}1
	{Á}{{\'A}}1
	{é}{{\'e}}1
	{É}{{\'E}}1
	{í}{{\'i}}1
	{Í}{{\'Í}}1
	{ó}{{\'o}}1
	{Ó}{{\'O}}1
	{ú}{{\'u}}1
	{Ú}{{\'U}}1
	{ü}{{\"u}}1
	{~}{{\textasciitilde}}1
	{ñ}{{\~n}}1
	{Ñ}{{\~N}}1,
	frame=tb,
	language=scala,
	aboveskip=3mm,
	belowskip=3mm,
	showstringspaces=false,
	columns=flexible,
	basicstyle={\footnotesize\ttfamily},
	numbers=none,
	numberstyle=\tiny\color{gray},
	keywordstyle=\color{blue}\bfseries,
	commentstyle=\color{dkgreen},
	stringstyle=\color{mauve},
	frame=single,
	breaklines=true,
	breakatwhitespace=true,
	keepspaces=true,
	tabsize=3
}

\usepackage{graphicx}
\graphicspath{{./img/}}

\usepackage{csquotes}
\usepackage[
	backend=bibtex,
	bibencoding=ascii,
	style=numeric,
	sorting=none
]{biblatex}
\bibliography{ia-tp2}

\title{Optimización de frecuencias de colectivos utilizando algoritmos genéticos}
\author{Cristian García \and Tomás Gropper \and Federico López Cañas \and Rodrigo López Dato \and Uriel Petasny \and Matías Petrone}

\begin{document}

\maketitle
\begin{abstract}
Buscamos optimizar las frecuencias con las que deben partir colectivos urbanos para minimizar el costo total, la cantidad de pasajeros que viajan parados, y la cantidad máxima de pasajeros esperando a la vez.
Para resolver este problema, implementamos un algoritmo genético utilizando una biblioteca propia en Scala, contemplando un modelo simplificado de la realidad.
\end{abstract}


\section{Introducción y modelización}

El problema que buscamos resolver es el de optimizar con qué frecuencias deben partir colectivos urbanos para minimizar las siguientes tres variables:

\begin{itemize}
	\item Costo total
	\item Cantidad de pasajeros que viajan parados
	\item Máxima cantidad de pasajeros esperando a la vez
\end{itemize}

Se considera que todas las variables tienen el mismo peso a la hora de comparar posibles soluciones.

Existen tres turnos: mañana, tarde y noche.
En cada turno viajan distintas cantidades de pasajeros, que modelamos mediante la cantidad de pasajeros que llegan por hora a la parada.

Para simplificar el modelo, tuvimos en cuenta varios supuestos.
Existe una única parada y un único destino; todos los pasajeros suben y descienden en las mismas paradas.
Cada colectivo tarda exactamente el mismo tiempo en realizar el recorrido.
Todos los colectivos son idénticos en cuanto a capacidad, tanto para pasajeros sentados como parados.
El costo de cada viaje es fijo, y no se consideran los ingresos por los boletos de los pasajeros.
Cada turno empieza "desde cero", es decir, al principio de cada turno se considera que todos los colectivos están listos para salir desde la parada.
Por último, consideramos que la frecuencia con la que llegan pasajeros es exacta, es decir, cada hora llega exactamente la cantidad especificada de pasajeros a la parada.

A partir de estos supuestos, definimos las siguientes variables de contexto:

\begin{itemize}
	\item Cantidad de colectivos
	\item Cantidad de asientos por colectivo
	\item Capacidad para pasajeros parados
	\item Tiempo de recorrido en minutos
	\item Costo de recorrido
\end{itemize}

\subsection{Implementación}

Elegimos implementar todo el código en Scala por conocimientos previos, la expresividad del lenguaje y la funcionalidad disponible en la biblioteca standard.
Scala también nos permite utilizar Scala.js\cite{scalajs} para facilitar la distribución de un ejecutable mediante una página web en lugar de un paquete instalable.

Como oportunidad de aprendizaje, optamos por crear una biblioteca de algoritmos genéticos con los operadores utilizados en este trabajo llamada \texttt{genetics}\cite{genetics}.

\section{Simulación}

Dado un turno, la cantidad de pasajeros a transportar por hora y las variables del contexto, resulta poco práctico, quizás imposible, encontrar una función\footnote{Nuestra implementación particular es determinista por lo que podría considerarse una función, pero no existe una fórmula que permita calcular el resultado de una simulación sin realizar las iteraciones necesarias.} que devuelva los resultados deseados.
Por este motivo, elegimos calcularlo mediante una simulación dinámica.

La simulación se ejecuta para cada turno de manera independiente, recibiendo como entrada las variables de contexto y la frecuencia de pasajeros del turno actual.
Devuelve como resultados las tres variables de interés: costo total, cantidad de pasajeros que viajan parados, y máxima cantidad de pasajeros esperando a la vez.

El código de la simulación está en el método \texttt{simulateShift} del tipo \texttt{BusContext}, que define también las variables de contexto.

\section{Algoritmo genético}

A continuación describimos cada componente del algoritmo implementado en detalle.
La implementación abstracta de un algoritmo genético, que es adaptable a otros operadores, se encuentra en la clase \texttt{Genetic}.

\subsection{Definición del cromosoma}

Definimos un cromosoma a partir de tres genes, cada uno representando la demora en minutos antes de enviar un colectivo en cada turno.
Por ejemplo, el cromosoma $(1,2,3)$ indica que cada colectivo en el turno noche realiza su viaje y espera tres minutos antes de llevar pasajeros nuevamente.

\subsection{Operadores genéticos}

\paragraph{Población inicial} Generación aleatoria de cromosomas de tamaño fijo, teniendo en cuenta valores máximos para cada frecuencia.
Por ejemplo:

\lstinputlisting{src/randomGene.scala}

\paragraph{Selección} Por ruleta. Se aplica repetidamente el algoritmo hasta llegar a un porcentaje configurable de selección relativo al tamaño de la población inicial.
Ver implementación en la clase \texttt{RouletteSelection}.

\paragraph{Cruzamiento} Simple.
Como estamos utilizando cromosomas con únicamente tres valores, elegimos siempre el segundo elemento como punto de cruzamiento (\texttt{xop} por \textit{\foreignlanguage{english}{crossover point}}), lo que evita devolver individuos duplicados al cruzar.
Se cruzan dos individuos al azar hasta llegar al tamaño de la población inicial.
Si se generan más individuos que los necesarios, se eliminan.
Ver implementación en la clase \texttt{OnePointCrossover}.

\paragraph{Mutación} Simple, de probabilidad constante. Se permite configurar el porcentaje de mutación.
Ver implementación en la clase \texttt{IntegerMutation}.

\subsection{Función de aptitud}

Como deseamos minimizar las tres variables resultantes de la simulación y las variables tienen el mismo peso, definimos la función de aptitud para esta terna de valores:

\[ f(c, p, m) = \frac{100000}{c + p + m} \]

donde $c$, $p$ y $m$ son costo total, pasajeros parados y máxima cantidad de pasajeros esperando respectivamente.
El numerador es 100000 para evitar errores de redondeo por usar números muy pequeños.

En nuestra implementación, los cromosomas son objetos de tipo \texttt{BusFrequency} que tienen un método \texttt{fitness}.
Este método ejecuta la simulación para cada turno, reúne los resultados en un objeto de tipo \texttt{SimulationResult} y calcula la aptitud utilizando la fórmula anterior.
El código relevante es el siguiente:

\lstinputlisting{src/AptitudeFunction.scala}

\section{Resultados}

En todos los casos, los parámetros de simulación utilizados fueron los siguientes:

\begin{itemize}
	\item Cantidad de colectivos: 10
	\item Asientos por colectivo: 30
	\item Capacidad para pasajeros parados: 15
	\item Duración del viaje: 60 minutos
	\item Costo del viaje: \$50
	\item Pasajeros por hora en cada turno: 500, 300 y 200
\end{itemize}

La población inicial en cada caso se generó tomando intervalos entre arribos aleatorios entre 0 y 60 minutos.

\newpage

\newgeometry{left=3cm,right=3cm,top=1cm,bottom=2cm}

\subsection{Corrida 1}

\subsubsection*{Parámetros aplicados}

\begin{multicols}{2}
\begin{itemize}
	\item Tamaño de población: 300
	\item Probabilidad de mutación: 1\%
	\item Porcentaje de selección: 80\%
	\item Cantidad de iteraciones: 100
\end{itemize}
\end{multicols}

\paragraph*{Mejor individuo}

$(0, 12, 0)$ con valor de aptitud 242.

\paragraph*{Mejor individuo en última población}

$(0, 22, 4)$ con valor de aptitud 121.

\subsubsection*{Evolución de aptitud media}

\includegraphics{data/img/evolution-1}

\subsubsection*{Histograma de población final}

\includegraphics{data/img/fitness-1}

\subsection{Corrida 2}

\subsubsection*{Parámetros aplicados}

\begin{multicols}{2}
\begin{itemize}
	\item Tamaño de población: 100
	\item Probabilidad de mutación: 1\%
	\item Porcentaje de selección: 80\%
	\item Cantidad de iteraciones: 2000
\end{itemize}
\end{multicols}

\subsubsection*{Mejor individuo}

$(0, 0, 17)$ con valor de aptitud 239, en última población.

\subsubsection*{Evolución de aptitud media}

\includegraphics{data/img/evolution-2}

\subsubsection*{Histograma de población final}

\includegraphics{data/img/fitness-2}

\subsection{Corrida 3}

\subsubsection*{Parámetros aplicados}

\begin{multicols}{2}
\begin{itemize}
	\item Tamaño de población: 100
	\item Probabilidad de mutación: 90\%
	\item Porcentaje de selección: 20\%
	\item Cantidad de iteraciones: 2000
\end{itemize}
\end{multicols}

\subsubsection*{Mejor resultado}

$(0, 1, 0)$ con valor de aptitud 249, en última población.

\subsubsection*{Evolución de aptitud media}

\includegraphics{data/img/evolution-3}

\subsubsection*{Histograma de población final}

\includegraphics{data/img/fitness-3}

\restoregeometry

\section{Análisis de resultados}

Podemos observar que las tres corridas obtuvieron resultados similares en cuanto al mejor individuo, sin embargo es interesante analizar los resultados enteros de cada corrida por separado.

La corrida 1 utiliza una población grande pero con una cantidad baja de iteraciones.
Llegó a un resultado alto por azar, pero la aptitud media final es la mas baja de todas las corridas.
Esto se debe a que nuestro algoritmo requiere de más iteraciones para converger en un máximo.

En cambio, la corrida 2 utiliza una población más pequeña con 20 veces más iteraciones. En este caso podemos observar claramente la convergencia de la aptitud media a un valor cercano a 230, no muy lejos del mejor individuo de toda la corrida.
Sin embargo, viendo el histograma de la población, se ve que existen dos grupos claramente marcados de aptitudes.
Analizando los datos completos pudimos observar que existen numerosos resultados duplicados en la población final, posiblemente debido a nuestra elección de método de cruzamiento.

Finalmente, la última corrida también utiliza una población pequeña con 2000 iteraciones, pero utiliza un porcentaje de selección más restrictivo y una alta probabilidad de mutación.
Esto se puede observar inmediatamente en el gráfico de evolución; no se nota una convergencia clara como en el caso de la segunda corrida.
También se observan grupos aislados de aptitudes en el histograma, causados por la duplicación de cromosomas a lo largo de tantas iteraciones.

\section{Conclusiones}

Habiendo implementado todos los operadores, pudimos observar la complejidad que implica diseñar cada uno y la cantidad de decisiones que deben tomarse para llegar a un resultado satisfactorio.
Concluimos que si tuviésemos que implementar un algoritmo genético para un problema real, no implementaríamos nuestra propia biblioteca.

Para mejorar nuestro algoritmo, deberíamos optimizar la implementación para poder hacer corridas en tiempos razonables con cientos de miles de individuos e iteraciones.
Para esto sería necesario eliminar los cromosomas duplicados.


\printbibliography

\end{document}